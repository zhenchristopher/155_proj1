\documentclass[11pt]{article}
\usepackage{amsmath}
\usepackage{amssymb}
\usepackage[margin=2.0cm]{geometry}
\usepackage{parskip}
\usepackage{empheq}
\usepackage{graphicx}
\usepackage{epstopdf}
\usepackage{braket}
\usepackage{cancel}
\usepackage[english]{babel}
\usepackage[utf8]{inputenc}
\usepackage{fancyhdr}
\usepackage{boldline}
\usepackage{lastpage} % used to get the page 4 / 20 behavior...
\DeclareGraphicsExtensions{.eps,.pdf,.png,.jpg}
\DeclareMathOperator{\arccosh}{arccosh}
\DeclareMathOperator{\Tr}{Tr}
\DeclareMathOperator{\Var}{Var}
\newcommand{\pd}[2][]{\frac{\partial#1}{\partial#2}}
\newcommand{\boldline}[1]{\underline{\textbf{#1}}}

\pagestyle{fancy}
\fancyhf{}
\lhead{CS 155 Kaggle Competition Report | 2/11/2018}
\rhead{\textbf{C. Zhen, David Wang, P. Vasireddy}}
\rfoot{Page \thepage \ of \pageref{LastPage}}

\title{CS 155 Kaggle Competition Report}
\author{Christopher Zhen, David Wang, and Praful Vasireddy}

\begin{document}
	
	\maketitle
	
	\pagestyle{fancy}
	
	% LaTeX is simple if you have a good template to work with! To use this document, simply fill in your text where we have indicated. To write mathematical notation in a fancy style, just write the notation inside enclosing $dollar signs$.
	
	% For example:
	% $y = x^2 + 2x + 1$
	
	% For help with LaTeX, please feel free to see a TA!
	
	
	
	\section{Introduction}
	\medskip
	\begin{itemize}
		
		\item \boldline{Group members} \\
		Christopher Zhen, David Wang, and Praful Vasireddy.
		
		\item \boldline{Team name} \\
		trust the process
		
		\item \boldline{Division of labour} \\
		Praful Vasireddy- tuned hyperparameters and tested gradient boosting model.
		
	\end{itemize}
	
	
	
	\section{Overview}
	\medskip
	\begin{itemize}
		
		\item \boldline{Models and techniques tried}
		\begin{itemize}
			% Insert text here. Bullet points can be made using '\item'. Models and techniques should be bolded using '\textbf{}'.
			\item \textbf{Random Forest:} Bullet text.
			\item \textbf{Neural Network:} Bullet text.
			\item \textbf{Gradient Boosting (xgboost):} Used hyperopt package for optimization of hyperparameters, tried tf-idf normalization for data, and tried different cross validation sizes.
			\item \textbf{Logistic Regression:} Bullet text.
			\item \textbf{Naive Bayes:} Bullet text.
			\item \textbf{Ensembling:} Bullet text.
		\end{itemize}
		
		\item \boldline{Work timeline}
		\begin{itemize}
			% Insert text here. Bullet points can be made using '\item'.
			\item \textbf{Bullet:} Bullet text.
		\end{itemize}
		
	\end{itemize}
	
	
	
	\section{Approach}
	\medskip
	\begin{itemize}
		
		\item \boldline{Data processing and manipulation}
		\begin{itemize}
			% Insert text here. Bullet points can be made using '\item'.
			\item \textbf{Bullet:} Bullet text.
		\end{itemize}
		
		\item \boldline{Details of models and techniques}
		\begin{itemize}
			% Insert text here. Bullet points can be made using '\item'.
			\item \textbf{Bullet:} Bullet text.
			
			% If you would like to insert a figure, you can just use the following five lines, replacing the image path with your own and the caption with a 1-2 sentence description of what the image is and how it is relevant or useful.
			\begin{center}
				\includegraphics[width = 0.6\textwidth]{3g}\\
				Figure 1: Schematic for studying the behavior of the low pass filter.
			\end{center}
			
		\end{itemize}
		
	\end{itemize}
	
	
	
	\section{Model Selection}
	\medskip
	\begin{itemize}
		
		\item \boldline{Scoring} \\
		% Insert text here.
		
		\item \boldline{Validation and Test} \\
		% Insert text here.
		
	\end{itemize}
	
	
	
	\section{Conclusion}
	\medskip
	\begin{itemize}
		
		\item \boldline{Discoveries} \\
		% Insert text here.
		
		\item \boldline{Challenges} \\
		% Insert text here.
		
		\item \boldline{Concluding Remarks} \\
		% Insert text here.
		
	\end{itemize}
	
	
	
\end{document}
	